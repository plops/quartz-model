\documentclass[twocolumn,10pt,DIV18]{scrartcl}
\usepackage{graphicx}
\usepackage[ngerman]{babel}
\usepackage{hyperref}
\usepackage{units}
\usepackage{amsmath}
\usepackage{verbatim}

\renewcommand{\v}[1]{{\mathbf #1}} % bold vector
\newcommand{\T}[1]{{\bar #1}} % tensor
\newcommand{\E}{\v E}
\newcommand{\eps}{\varepsilon}
\newcommand{\D}{\v D}
\newcommand{\B}{\v B}
\renewcommand{\H}{\v H}
\newcommand{\hk}{\hat{\v k}} % Einheitswellenvektor
\renewcommand{\r}{\v r}
\renewcommand{\[}{\left[}
\renewcommand{\]}{\right]}
\renewcommand{\(}{\left(}
\renewcommand{\)}{\right)}
% Excerpt von Stephen C. McClain 1992
\title{Quartz -- Modellierung optischer Eigenschaften und Simulation}
\author{Martin Kielhorn}
\date{30. September 2009}

\begin{document}
  \maketitle 

  Um die Ausbreitung von Licht in anderen Medien als Vakuum zu
  beschreiben, muss man zus\"atzlich zu den Maxwell Gleichungen die
  Materialgleichungen des Mediums kennen. Diese Gleichungen geben an,
  wie die elektromagnetischen Feldvektoren $\E$, $\D$, $\B$ und $\H$
  im Medium miteinander wechselwirken. Beispielsweise kann $\E$ freie
  Ladungen bewegen. Das Magnetfeld dieses Stroms \"andert daraufhin $\B$.

  Das wichtigste Material in der Optik ist Glas. Es ist (f\"ur
  optische Wellenl\"angen) linear, isotrop, homogen, nicht magnetisch,
  nicht optisch aktiv und nicht absorbierend. F\"ur diesen Fall sehen
  die Materialgleichungen einfach aus:
  \begin{align}
    \D&=\eps\E,\\
    \B&=\H.
  \end{align}
  Die Dielektrizit\"atskonstante ist ein frequenzabh\"angiger Skalar.
  
  In einem anisotropen Medium haben die Elektronen bestimmte
  Vorzugsrichtungen. Die erste Materialgleichung muss dann mit einem
  Tensor $\T\eps$ beschrieben werden:
  \begin{align}
    \D=\T \eps\E.
  \end{align}
  Quartz ist ein einachsiger, trigonaler Kristall mit einer
  Spiegelasymmetrie. Der Quartzkristall ist invariant unter
  $120^\circ$ oder $240^\circ$ Drehungen um seine Kristallachse $\hat
  c$. Diese Rotationssymmetrie spiegelt sich auch im dielektrischen
  Tensor $\T\eps$ wieder:
  \begin{align}
    \T\eps=\begin{pmatrix}
    \eps_o&0&0\\
    0&\eps_o&0\\
    0&0&\eps_e
    \end{pmatrix}.
  \end{align}
  Die fehlende Spiegelsymmetrie bewirkt, dass im Kristall Spiralen
  entstehen. Der Kristall ist chiral und damit optisch aktiv. Die
  durch ein \"au\ss eres elektrisches Feld bewegten Ladungen werden im
  Kristallgitter auf helixf\"ormige Bahnen gezwungen. Dieser Strom
  ruft ein magnetisches Dipolmoment der Gr\"o\ss enordnung $\dot\E$
  hervor. Analog entsteht in einem Magnetfeld eine elektrische
  Polarization der Gr\"o\ss enordnugn $\dot\H$.
  
  Der Quartz Kristall ist optisch aktiv, anisotrop und nicht
  magnetisch (d.h. $\T\mu$ wird der Einheitstensor $\T 1$). Deshalb
  gelten in ihm die Materialgleichungen:
  \begin{align}
    \D&=\T\eps\E+\T\chi\H,\\
    \B&=\H+\T\chi\E.
  \end{align}
  Wobei der Tensor $\T\chi$ auf Grund der Dreifachsymmetrie eine zu
  $\T\eps$ analoge Form aufweist. Mit der Zeitabh\"angigkeit
  $\exp(-i\omega t)$ der Felder kann der neue Tensor $\T
  G=\omega\T\chi$ eingef\"uhrt werden:
  \begin{align}
    \T G=\begin{pmatrix}
    g_o&0&0\\
    0&g_o&0\\
    0&0&g_e
    \end{pmatrix}.
  \end{align}
  Damit ergibt sich f\"ur die Materialgleichung in Einsteinscher
  Summennotation:
  \begin{align}
    D_k&=\eps_{jk}E_j+iG_{jk}H_j,\\
    B_k&=\delta_{jk}H_j-iG_{jk}E_j,
  \end{align}
  und ausgeschrieben:
  \begin{align}
    \D=\begin{pmatrix}
    \eps_o E_x+ig_oH_x\\
    \eps_o E_y+ig_oH_y\\
    \eps_e E_z+ig_eH_z
    \end{pmatrix},\quad
    \B=\H-i\begin{pmatrix}
    g_oE_x\\
    g_oE_y\\
    g_eE_z
    \end{pmatrix}.
  \end{align}
  \section{Eigenwertgleichungen}
  Wenn man die Materialgleichungen in die Maxwellgleichungen einsetzt,
  erh\"alt man eine Eigenwertgleichung f\"ur das elektrische
  Feld. Diese fa\ss t den Zusammenhang zwischen Wellenvektor,
  Brechzahl und Polarisationszustand kompakt zusammen. Man
  beginnt man mit einer ebenen, monochromatischen Welle:
  \begin{align}
    \E(\r,t)=\textrm{Re}\[\E_0\exp\(i\(\frac{n\omega}{c}\hk\r-\omega t\)\)\]
  \end{align}
  und eliminiert alle Felder bis auf $\E$.  Das Ergebnis ist die
  Eigenwertgleichung
  \begin{align}\label{eq:ewgl}
    \T\eps+\Bigl[n\underbrace{\begin{pmatrix}0&-k_z&k_y\\
    k_z&0&-k_x\\
    -k_y&k_x&0\end{pmatrix}}_{\T K}+i\T G\Bigr]^2\,\E=0.
  \end{align}
  Hierbei ist $\T K$ ein Kreuzproduktoperator, dessen Komponenten vom
  Ausbreitungsvektor $\hk$ abgeleitet sind.  F\"ur eine nichttriviale
  L\"osung von $\E$ muss die Determinante der gesamten Matrix auf der
  linken Seite verschwinden. Um zur Relation zwischen dem
  Brechungsindex und der Ausbreitungsrichtung zu kommenm, sei die
  Kristallachse $\hat c$ entlang $\v e_z$ ausgerichtet. In diesem Fall
  nehmen die Tensoren $\T\eps$ und $\T G$ diagonale Form an.
  Weiterhin erfolge die Ausbreitung in der yz-Ebene. Der
  Ausbreitungsvektor kann als $\hk=(0,\sin\theta,\cos\theta)^T$
  geschrieben werden. Diese Vereinfachungen darf man vornehmen, denn
  das Problem ist symmetrisch bez\"uglich der Rotation um die $\hat
  c$-Achse. Mit diesen Umformungen wird $(n\T K+i\T G)^2$ zu
  \begin{align*}
    (n\T K+i\T G)^2=
    \begin{pmatrix}
      i g_o      &         -nc         &       ns       \\
      nc          &      i g_o         &       0       \\
      -ns         &         0          &     i g_e 
    \end{pmatrix}^2=\\
    \begin{pmatrix}
      -g_o^2-n^2c^2-n^2s^2  &  -2ig_onc     &  ig_ons+ig_ens \\
      2ig_onc               & -n^2c^2-g_o^2 & n^2sc \\
      -ig_ons-ig_ens        & n^2sc         & -n^2s^2-g_e^2
    \end{pmatrix}.
  \end{align*}
  mit $c=\cos\theta$, $s=\sin\theta$.  Die gesamte Matrix in der
  Eigenwertgleichung wird:
  \begin{align}
    \!\!\!\begin{pmatrix}
      \eps_o'-n^2    &  -2ing_oc           &   in(g_o+g_e)s       \\
      2ing_o c       &  \eps_o'-n^2c^2     &   n^2sc              \\
      -in(g_o+g_e)s  &  n^2sc              &   \eps_e'-n^2s^2
    \end{pmatrix}
  \end{align}
  mit $\eps_o'=\eps_o-g_o^2$ und $\eps_e'=\eps_e-g_e^2$.  Das
  Nullsetzen der Determinante liefert eine quadratische Gleichung
  f\"ur $n^2$:
  \begin{align}
    (n^2)^2(e+w)-n^2\(\eps_o'(\eps_o'+e)+v\)+\eps_o'^2\eps_e'=0
  \end{align}
  mit
  \begin{align}
    e&=c^2\eps_e'+s^2\eps_o',\\
    v&=4\eps_e'g_o^2c^2+\eps_o'(g_o+g_e)^2s^2, \\
    w&=(g_o-g_e)^2s^2c^2.
  \end{align}
  Die zwei L\"osungen f\"ur $n^2$ sind
  \begin{align}\label{eq:n2eo}
    (n^2)_{e/o}=\frac{\eps_o'(\eps_e'+e)+v
      \pm\sqrt{q}}{2(e+w)}
  \end{align}
  mit
  \begin{align}
    q=\eps_o'^2(\eps_o'+e)^2+2\eps_o'v(\eps_e'+e)+v^2-4(e+w)\eps_o'^2\eps_e'.
  \end{align}
  F\"ur $n$ liefert Gleichung \eqref{eq:n2eo} vier L\"osungen. Zwei
  davon sind negativ. Sie entsprechen dem negativem Ausbreitungsvektor
  und werden hier ignoriert. Die anderen beiden sind f\"ur den
  ordentlichen ($n_o$) und den au\ss erordentlichen Strahl ($n_e$).

  Um die zu einer Brechzahl geh\"orige Polarisationsmode $\hat\E$ zu
  ermitteln, wird der entsprechende Index $n$ in die
  Eigenwertgleichung \eqref{eq:ewgl} eingesetzt und die normierte Mode
  $\hat\E$ bestimmt.  F\"ur Ausbreitung entlang der optischen Achse
  ($\theta=0$) ergeben sich:
  \begin{align} %FIXME check right and left
    n_r&=n_e=\sqrt{\eps_o}+g_o, &\hat\E_e&=\frac{1}{\sqrt{2}}(1,-i,0)^T,\\
    n_l&=n_o=\sqrt{\eps_o}-g_o, &\hat\E_o&=\frac{1}{\sqrt{2}}(1,i,0)^T.
  \end{align}
  Die optische Aktivit\"at bewirkt im rechtsh\"andigen Quartz, dass
  rechtsdrehendes zirkular polarisiertes Licht langsamer l\"auft als
  linksdrehendes.  Messwerte f\"ur beide Brechzahlen bei
  \unit[762]{nm} sind $n_r=1.53920$ und $n_l=1.53914$. Damit k\"onnen
  \"uber die beiden obigen Gleichung $\eps_o$ und $g_o$ bestimmt
  werden:
  \begin{align}
    \eps_o&=(n_r+n_l)^2/4=\underbar{2.36904}4,\\
    \Delta\eps_o&=\Delta n_r \frac{n_r+n_l}{2}+\Delta n_l \frac{n_r+n_l}{2}=0.000015,\\
    g_o&=(n_r-n_l)/2=\underbar{0.000030}0,\\
    \Delta g_o &= \Delta n_r/2 + \Delta n_l/2 = 0.000005
  \end{align}
  Hierbei gehe ich von einem halben Skalenteil Fehler $\Delta
  n_{l/r}=0.000005$ in den Messwerten aus.
\end{document} 